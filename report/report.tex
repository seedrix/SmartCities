% This is samplepaper.tex, a sample chapter demonstrating the
% LLNCS macro package for Springer Computer Science proceedings;
% Version 2.20 of 2017/10/04
%
\documentclass[runningheads]{llncs}
%
\usepackage{graphicx}

\begin{document}
%
\title{Come Up with a Title for Your Project}

\author{Group ID: First Author \and
Second Author \and
Thomas Wangler}

\institute{Service Computing Department, IAAS, University of Stuttgart
\email{firstname.lastname@uni-stuttgart.de}}
%
\maketitle              % typeset the header of the contribution
%
\begin{abstract}
The abstract should briefly summarize the contents of the report in
150--250 words. 

\keywords{First keyword  \and Second keyword \and Another keyword.}
\end{abstract}
%
%
%
\section{System Introduction}
In the current context of the Coronavirus monitoring and controlling the room occupancy as well as the streams of people is an important task to reduce the infection rate. Therefore, it can be used to reopen facilities while minimizing the risks. For this purpose, our system counts the number of people in specific rooms, evaluates those numbers and reacts to crowding by specifying alternative ways or blocking the entry to those rooms.

Concrete use cases are, e.g., supermarkets or shopping malls. In this context our system regulates the number of customers visiting the shop at any time. Based on collected data, customers can be notified about times with a unusual high or low number of customers. 

\section{System Analysis}
Describe the user requirements of your system.

\section{System Architecture Design}
Describe and provide a design of the architecture of your system.

\section{System Implementation}
Describe the implementation of your system. This section is only relevant for the report and should be omitted for the project description. 

\section{Discussion and Conclusions}
Here you can discuss some interesting points or limitations of your system and conclude the report.

%
% ---- Bibliography ----
%
\bibliographystyle{splncs04}
\bibliography{mybib}

All links were last followed on April 17, 2020.

\end{document}
